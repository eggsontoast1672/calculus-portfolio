\documentclass{article}

% I want this stuff before the rest of the preamble because it's the stuff that
% I'll most likely be looking for at the top of the file.
\title{My First LaTeX Document}
\author{Paul Zupan}
\date{September 16, 2023}

\usepackage{amsmath}
\usepackage{amssymb}
\usepackage{enumitem}
\usepackage{pgfplots}

\pgfplotsset{compat=1.18}

\begin{document}

\maketitle

\section{Limits and Continuity}

\subsection{Introducing Calculus: Can Change Occur at an Instant?}

\begin{tikzpicture}[
    declare function={
        func(\x) =
            and(\x >= 0, \x <= 6) * (5 * \x + 10) +
            and(\x > 6, \x <= 10) * (40) +
            and(\x > 10, \x <= 12) * (10 * \x - 60);
    }
]
    \begin{axis}[xmin=0, xmax=12, ymin=0, ymax=60]
        \addplot[domain=0:12]{func(x)};
    \end{axis}
\end{tikzpicture}

A particle is moving on the $x$-axis and the position of the particle at time
$t$ is given by $x(t)$, whose graph is shown above. Which of the following is
the best estimate for the speed of the particle at time $t = 4$?

\begin{enumerate}[label=\Alph*]
    \item 0
    \item 5
    \item $\frac{15}{2}$
    \item 10
\end{enumerate}

First, we know that the graph represents the speed of the particle for a time
$t$. That is another way of saying that it represents the rate of change of the
position of the particle with respect to time at a time $t$, and since $t$
represents a point in time, we could say that we are being asked to find
$\frac{d}{dt} x(t)$ for $t = 4$.

The derivative is just the slope, so we need to find the slope of $x(t)$ for $x
= 4$. This can be done by taking two points on the first portion of the graph,
that is to say any two values of $x(t)$ such that $0 \leq t \leq 6$. Here, I
choose to use (2, 20) and (6, 40).

The slope of the line that passes through those two points, given by $m$, can
be calculated as follows:

\begin{align}
    m &= \frac{40 - 20}{6 - 2} \\
      &= \frac{20}{4} \\
      &= 5
\end{align}

Now, it can be plainly seen that the answer is B.

$\blacksquare$

\subsection{Defining Limits and Using Limit Notation}

\begin{tikzpicture}[
    declare function={
        func(\x) =
            (\x < 2) * (2) +
            (\x == 2) * 1 +
            (\x > 2) * \x;
    }
]
    \begin{axis}[xmin=-1, xmax=4, ymin=-2, ymax=4]
        \addplot[domain=-1:4]{func(x)};
    \end{axis}
\end{tikzpicture}

The graph of the function $f$ is shown above. What is $lim_{x \to 2} f(x)$?

\begin{enumerate}[label=\Alph*]
    \item 0
    \item 1
    \item 2
    \item The limit does not exist
\end{enumerate}

One way to evaluate a limit expression is to compare the values of the left and
right hand limits. That is the strategy that we will use here. In the absence
of an accompanying equation, we must make an assumption. I will assume that
$y(x)$ is represented by the following piecewise-define equation:

\begin{math}
    f(x) = \begin{cases}
        2 \text{ if } x < 2 \\
        1 \text{ if } x = 2 \\
        x \text{ if } x > 2
    \end{cases}
\end{math}

Now, we are free to take the limit from the left hand side:

\begin{align}
    \lim_{x \to 2 ^ -} f(x)
        &= \lim_{x \to 2 ^ -} 2 \\
        &= 2
\end{align}

Next, the right hand side:

\begin{align}
    \lim_{x \to 2 ^ +} f(x)
        &= \lim_{x \to 2 ^ +} x \\
        &= 2
\end{align}

Both the left and right hand limits evaluate to 2, therefore $\lim_{x \to 2}
f(x) = 2$.

$\blacksquare$

\subsection{Estimating Limit Values from Graphs}

Add graph here.

The graph of the function $h$ is shown above. What is $\lim_{x \to 4} h(x)$?

\begin{enumerate}[label=\Alph*]
    \item -1
    \item 2
    \item 5
    \item nonexistent
\end{enumerate}

Here, we can use the same technique as the last problem (see section 1.2). We
must make another assumption about the equation that the graph above
represents, as an equation was not provided.

Actually, maybe it would be better to use an epsilon delta proof here.

\end{document}
